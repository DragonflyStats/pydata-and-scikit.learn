\documentclass[MASTER.tex]{subfiles} 
\begin{document} 
	
%===========================================================%
\begin{frame}
	\frametitle{Versions of Python}
	\LARGE
	Two Main Versions of Python
	\begin{itemize}
		\item Version 2.7
		\item Version 3
	\end{itemize}
	
\end{frame}
%===========================================================%
\begin{frame}
\frametitle{Python Coding Conventions}
\large
There are a number of common practices which can be adopted to produce Python code which looks
more like code found in other modules:
\begin{itemize}
	\item Use 4 spaces to indent blocks – avoid using tab, except when an editor automatically converts tabs
	to 4 spaces
	\item Avoid more than 4 levels of nesting, if possible
	\item Limit lines to 79 characters. The $\backslash$ symbol can be used to break long lines
	219
	\item Use two blank lines to separate functions, and one to separate logical sections in a function.
\end{itemize}
	
\end{frame}
%===========================================================%
\begin{frame}
\frametitle{Python Coding Conventions}
\large
\begin{itemize}	
	\item Use ASCII mode in text editors, not UTF-8
	\item One module per import line
	\item Avoid from module \texttt{import $\ast$} (for any module). Use either \texttt{from module import func1, func2} or
	\texttt{import module as shortname}.
	\item Follow the NumPy guidelines for documenting functions
\end{itemize}

% More suggestions can be found in PEP8.
\end{frame}
%===========================================================%
\end{document}