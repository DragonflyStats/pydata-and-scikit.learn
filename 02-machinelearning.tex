\documentclass[MASTER.tex]{subfiles} 
\begin{document} 
	
	
	\begin{frame}
		\huge
		\[ \mbox{Machine Learning with Python} \]
	\end{frame}
	
	%===========================================================%%=======================================================================%
\begin{frame}
\Large
\textbf{MachineLearning}

Machine Learning is a discipline involving algorithms designed to find patterns in and make predictions about data. It is nearly ubiquitous in our world today, and used in everything from web searches to financial forecasts to studies of the nature of the Universe. This tutorial will offer an introduction to scikit-learn, a python machine learning package, and to the central concepts of Machine Learning. 

\end{frame}
%=======================================================================%
\begin{frame}
\Large	\textbf{MachineLearning}
	We will introduce the basic categories of learning problems and how to implement them using scikit-learn. From this foundation, we will explore practical examples of machine learning using real-world data, from handwriting analysis to automated classification of astronomical images.

\end{frame}

\end{document}