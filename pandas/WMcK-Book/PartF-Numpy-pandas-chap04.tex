
\documentclass[11pt]{article} % use larger type; default would be 10pt

\usepackage[utf8]{inputenc} 
\usepackage{geometry} % to change the page dimensions
\geometry{a4paper} 
\usepackage{graphicx} 
\usepackage{booktabs} % for much better looking tables
\usepackage{array} % for better arrays (eg matrices) in maths
\usepackage{paralist} % very flexible & customisable lists (eg. enumerate/itemize, etc.)
\usepackage{verbatim} % adds environment for commenting out blocks of text & for better verbatim
\usepackage{subfig} 
\usepackage{framed}
\usepackage{subfiles}
\usepackage{fancyhdr} % This should be set AFTER setting up the page geometry
\pagestyle{fancy} % options: empty , plain , fancy
\renewcommand{\headrulewidth}{0pt} % customise the layout...
\lhead{}\chead{Data Analysis with Python}\rhead{}
\lfoot{}\cfoot{\thepage}\rfoot{}
%--------------------------------------------------------------------------------------------%
\usepackage{sectsty}
\allsectionsfont{\sffamily\mdseries\upshape} 
\usepackage[nottoc,notlof,notlot]{tocbibind} % Put the bibliography in the ToC
\usepackage[titles,subfigure]{tocloft} % Alter the style of the Table of Contents
\renewcommand{\cftsecfont}{\rmfamily\mdseries\upshape}
\renewcommand{\cftsecpagefont}{\rmfamily\mdseries\upshape} % No bold!
%--------------------------------------------------------------------------------------------%

\title{Brief Article}
\author{The Author}
%--------------------------------------------------------------------------------------------%

\begin{document}
\section{NumPy Basics: Arrays and Vectorized Computation}

NumPy, short for Numerical Python, is the fundamental package required for high
performance scientific computing and data analysis. It is the foundation on which
nearly all of the higher-level tools in this book are built. Here are some of the things it
provides:

\begin{itemize}
\item \texttt{ndarray}, a fast and space-efficient multidimensional array providing vectorized
arithmetic operations and sophisticated broadcasting capabilities
\item Standard mathematical functions for fast operations on entire arrays of data
without having to write loops
\item Tools for reading / writing array data to disk and working with memory-mapped
files
\item Linear algebra, random number generation, and Fourier transform capabilities
\item Tools for integrating code written in C, C++, and Fortran
\end{itemize}

\newpage
%----------------------------------------------------------------------------------%

\subsection{Basic Indexing and Slicing}
NumPy array indexing is a rich topic, as there are many ways you may want to select
a subset of your data or individual elements. One-dimensional arrays are simple; on
the surface they act similarly to Python lists.
%----------------------------------------------------------------------------------%
\subsection{Data Processing Using Arrays}

\begin{itemize}
\item Using NumPy arrays enables you to express many kinds of data processing tasks as
concise array expressions that might otherwise require writing loops. This practice of
replacing explicit loops with array expressions is commonly referred to as \textbf{vectorization
}.

\item In general, vectorized array operations will often be one or two (or more) orders
of magnitude faster than their pure Python equivalents, with the biggest impact in any
kind of numerical computations.
\end{itemize}
%----------------------------------------------------------------------------------%
\newpage
An ndarray is a NumPy array.
>>> x = np.array([1, 2, 3])
>>> type(x)
<type 'numpy.ndarray'>
The difference between np.ndarray and np.array is that the former is the actual type, while the latter is a flexible shorthand function for constructing arrays from data in other formats. The TypeError comes your use of np.array arguments to np.ndarray, which takes completely different arguments (see docstrings).

%----------------------%
what is the difference between ndarray and array in numpy? And where can I find the implementations in the numpy source code?

 np.array is just a convenience function to create an ndarray, it is not a class itself. 

You can also create an array using np.ndarray, but it is not the recommended way. 

From the docstring of np.ndarray: 


Arrays should be constructed using array, zeros or empty ... 
The parameters given here refer to a low-level method (ndarray(...)) for instantiating an array.
 
Most of the meat of the implementation is in C code, here in multiarray, but you can start looking at the ndarray interfaces here:

https://github.com/numpy/numpy/blob/master/numpy/core/numeric.py
%----------------------------------------------------------------------------------%
 


numpy.array is a function that returns a numpy.ndarray. There is no object type numpy.array.
 
%-----------------------------------%
%----------------------------------------------------------------------------------%
\end{document}


% - http://mikolalysenko.github.io/ndarray-presentation/
