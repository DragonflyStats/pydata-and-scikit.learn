\documentclass[KSmain.tex]{subfiles} 
\begin{document} 
\section{Python 2.7 vs. 3}
Python 2.7 is the final version of the Python 2.x line – all future development work will focus on Python 3.
It may seem strange to learn an “old” language. The reasons for using 2.7 are:
\begin{itemize}
\item There are more modules available for Python 2.7. While all of the core python modules are available
for both Python 2.7 and 3, some of the more esoteric modules are either only available for 2.7 or
have not been extensively tested in Python 3. Over time, many of these modules will be available for
Python 3, but they aren’t ready yet.
\item The language changes relevant for numerical computing are very small – and these notes explicitly
minimize these so that there should few changes needed to run against Python 3+ in the future
(ideally none).
\item Configuring and installing 2.7 is easier.
\item Anaconda defaults to 2.7 and the selection of packages available for Python 3 is limited.
\end{itemize}
Learning Python 3 has some advantages:
\begin{itemize}
\item No need to update in the future.
\item Some improved out-of-box behavior for numerical applications.
\end{itemize}

\subsection{Relevant Differences between Python 2.7 and 3}
Most differences between Python 2.7 and 3 are not important for using Python for analysis. 

%I will make three common assumptions which will allow 2.7 and 3 to be used
%interchangeable. The configuration instructions in the previous chapter for IPython will produce the expected behavior when run interactively. Note that these differences are important in stand-alone Python programs.
\subsubsection*{print}
print is a function used to display test in the console when running programs. In Python 2.7, \texttt{print} is a
keyword which behaves differently from other functions. In Python 3, \texttt{print} behaves like most functions.
The standard use in Python 2.7 is
\begin{framed}
	\begin{verbatim}
	print ’String to Print’
	\end{verbatim}
\end{framed}
while in Python 3 the standard use is
\begin{framed}
	\begin{verbatim}
	print(’String to Print’)
	\end{verbatim}
\end{framed}
which resembles calling a standard function. Python 2.7 contains a version of the Python 3 \texttt{print}, which
can be used in any program by including
\begin{framed}
	\begin{verbatim}
	from __future__ import print_function
	\end{verbatim}
\end{framed}
at the top of the file. 
%I prefer the Python 3 version of print, and so I assume that all programs will include this statement.
%-------------------------------------------------%
\subsubsection*{Division}
\begin{itemize}
	\item Python 3 changes the way integers are divided. In Python 2.7, the ratio of two integers was always an
	integer, and so results are truncated towards 0 if the result was fractional. For example, in Python 2.7, 9/5
	is 1. 
	\item Python 3 gracefully converts the result to a floating point number, and so in Python 3, 9/5 is 1.8. When
	working with numerical data, automatically converting ratios avoids some rare errors.
	\item Python 2.7 can use
	the Python 3 behavior by including
	\begin{framed}
		\begin{verbatim}
		from __future__ import division
		\end{verbatim}
	\end{framed}
	at the top of the program. We will assume that all relevant programs will include this statement.
\end{itemize}
%------------------------------------%
\subsubsection*{\texttt{range} and \texttt{xrange}}
It is often useful to generate a sequence of number for use when iterating over the some data. In Python
2.7, the best practice is to use the keyword \texttt{xrange} to do this, while in Python 3, this keyword has been
renamed \texttt{range}. I will always use \texttt{xrange} and so it is necessary to replace \texttt{xrange} with \texttt{range} if using Python
3.
%%------------------------------------%
%\subsubsection*{Unicode strings}
%Unicode is an industry standard for consistently encoding text. The computer alphabet was originally limited
%to 128 characters which is insufficient to contain the vast array of characters in all written languages.
%Unicode expands the possible space to be up to 231 characters (depending on encoding). Python 3 treats
%all strings as unicode unlike Python 2.7 where characters are a single byte, and unicode strings require the
%special syntax u’unicode string’ or unicode(’unicode string’). In practice this is unlikely to impact
%most numeric code written in Python except possibly when reading or writing data. If working in a language
%where characters outside of the standard but limited 128 character set are commonly encountered,
%it may be useful to use
%\begin{framed}
%	\begin{verbatim}
%	from __future__ import unicode_literals
%	\end{verbatim}
%\end{framed}
%to will help with future compatibility when moving to Python 3.
\end{document}