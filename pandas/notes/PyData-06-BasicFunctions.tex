\documentclass[KSmain.tex]{subfiles} 
\begin{document} 

\section{Basic Functions and Numerical Indexing}
% 6.3 Mathematics
% 6.2 Rounding and Precision
% 6.1 Moved to last
%=================================%
\subsection{Mathematical Functions}
\subsubsection*{\texttt{sum} and, \texttt{cumsum}}
\texttt{sum} sums elements in an array. By default, it will sum all elements in the array, and so the second argument
is normally used to provide the axis to use 
\begin{itemize}
\item 0 to sum down columns
\item 1 to sum across rows
\end{itemize}
\texttt{cumsum} produces
the cumulative sum of the values in the array, and is also usually used with the second argument to indicate
the axis to use.
%----------------%
\begin{framed}
\begin{verbatim}
>>> x = randn(3,4)
>>> x
array([[0.08542071, 2.05598312, 2.1114733  , 0.7986635 ],
       [0.17576066, 0.83327885, 0.64064119 , 0.25631728],
       [0.38226593, 1.09519101, 0.29416551 , 0.03059909]])
>>>
>>> sum(x) # all elements
0.62339964288008698
>>>
>>> sum(x, 0) # Down rows, 4 elements
array([0.6434473, 2.31789529, 1.76499762, 0.57294532])
>>>
>>> sum(x, 1) # Across columns, 3 elements
array([ 0.76873297, 
        0.23944028,
        1.15269233])
>>>
>>> cumsum(x,0) # Down rows
array([[0.08542071, 2.05598312, 2.1114733 , 0.7986635 ],
[0.26118137, 1.22270427, 1.47083211, 0.54234622],
[0.6434473 , 2.31789529, 1.76499762, 0.57294532]])
\end{verbatim}
\end{framed}
%----------------------%
\noindent \texttt{sum} and \texttt{cumsum} can both be used as function or as methods. When used as methods, the first input is the
axis so that \texttt{sum(x,0)} is the same as\texttt{ x.sum(0)}.
\newpage
\subsubsection*{\texttt{prod}, \texttt{cumprod}}
\texttt{prod} and \texttt{cumprod} behave similarly to \texttt{sum} and \texttt{cumsum} except that the product and cumulative product are
returned. As with \texttt{sum} and \texttt{cumsum}, \texttt{prod} and \texttt{cumprod} can be called as function or methods.

\subsubsection*{\texttt{diff}}
\texttt{diff} computes the finite difference of a vector (also array) and returns \textbf{\textit{n-1}} an element vector when used on
an \textbf{\textit{n}} element vector. 

\noindent diff operates on the last axis by default, and so diff(x) operates across columns and
returns \texttt{x[:,1:size(x,1)]x[:,:
size(x,1)1]}
for a 2-dimensional array. 

\texttt{diff} takes an optional keyword argument axis so that diff(x, axis=0) will operate across rows. \texttt{diff} can also be used to produce higher
order differences (e.g. double difference).
%----------------%
\begin{framed}
\begin{verbatim}>>> x= randn(3,4)
>>> x = randn(3,4)
>>> x
array([[0.08542071, 2.05598312, 2.1114733  , 0.7986635 ],
[0.17576066, 0.83327885, 0.64064119 , 0.25631728],
[0.38226593, 1.09519101, 0.29416551 , 0.03059909]])
>>>
>>> diff(x, axis=0)
array([[0.09033996, 2.88926197, 2.75211449,1.05498078],
[0.20650526, 1.92846986, 0.9348067 , 0.28691637]])
>>>
>>> diff(x, 2, axis=0) # Double difference, columnbycolumn
array([[0.11616531,
4.81773183,
3.68692119, 1.34189715]])
\end{verbatim}
\end{framed}
%----------------------%
\subsubsection*{\texttt{exp}}
\texttt{exp} returns the element-by-element exponential ($e^x$) for an array.
\subsubsection*{\texttt{log}}
\texttt{log} returns the element-by-element natural logarithm ($ln(x )$) for an array.
\subsubsection*{\texttt{log10}}
\texttt{log10} returns the element-by-element base-10 logarithm (\texttt{log10 (x )}) for an array.

%====================================%
\subsection{Rounding}
\subsubsection*{\texttt{around, round}}
around rounds to the nearest integer, or to a particular decimal place when called with two arguments.
\begin{framed}
\begin{verbatim}
>>> x = randn(3)
array([ 0.60675173, 0.3361189 , 0.56688485])
>>> around(x)
array([ 1., 0., 1.])
>>>
>>> around(x, 2)
array([ 0.61, 0.34, 0.57])
\end{verbatim}
\end{framed}
%----------------------%
\noindent \texttt{around} can also be used as a method on an ndarray – except that the method is named round. For example,
\texttt{x.round(2) }is identical to \texttt{around(x, 2)}. 

The change of names is needed to avoid conflicting with the Python built-in function \texttt{round}.
\newpage

\subsubsection*{\texttt{floor}}
\texttt{floor} rounds to the next smallest integer.
\begin{framed}
\begin{verbatim}
>>> x = randn(3)
array([ 0.60675173, 0.3361189, 0.56688485])
>>>
>>> floor(x) 
array([ 0., 1., 1.])
\end{verbatim}
\end{framed}
%----------------------%
\subsubsection*{\texttt{ceil}}
\texttt{ceil} rounds to the next largest integer.
%----------------%
\begin{framed}
\begin{verbatim}
>>> x = randn(3)
array([ 0.60675173, 0.3361189 , 0.56688485])
>>>
>>> ceil(x) 
array([ 1., 0., 0.])
\end{verbatim}
\end{framed}
%----------------------%
Note that the values returned are still floating points and so 0.
is the same as 0..

%========================================%
% Section 6.1


\subsection{Generating Arrays and Matrices}
\subsubsection*{\texttt{linspace}}
\texttt{linspace(l,u,n)} generates a set of n points uniformly spaced between l, a lower bound (inclusive) and u,
an upper bound (inclusive).
\begin{framed}
\begin{verbatim}
>>> x = linspace(0, 10, 11)
>>> x
array([ 0., 1., 2., 3., 4., 5., 6., 7., 8., 9., 10.])
\end{verbatim}
\end{framed}
%----------------------------------%
\subsubsection*{\texttt{logspace}}
\texttt{logspace(l,u,n)} produces a set of logarithmically spaced points between 10l and 10u . It is identical to
10**linspace(l,u,n).
%----------------------------------%
\subsubsection*{\texttt{arange}}
\begin{itemize}
\item \texttt{arange(l,u,s)} produces a set of points spaced by stepsize \textbf{\textit{s}} between \textbf{\textit{l}}, a lower bound (inclusive) and \textbf{\textit{u}}, an upper
bound (exclusive). 
\[  (l , l+s, l+2s , \ldots, u-s)\]
\begin{framed}
	\begin{verbatim}
	>>> x = arange(4, 10, 1.25)
	array([ 4. , 5.25, 6.5 , 7.75, 9. ])
	\end{verbatim}
\end{framed}

\item \texttt{arange} can be used with a single parameter, so that \texttt{arange(n)} is equivalent to
\texttt{arange(0,n,1)}.
\begin{framed}
	\begin{verbatim}
	>>> x = arange(11.0)
	array([ 0., 1., 2., 3., 4., 5., 6., 7., 8., 9., 10.])
	\end{verbatim}
\end{framed}
\item Note that \texttt{arange} will return integer data type if all inputs are integer.
\end{itemize}

\begin{framed}
\begin{verbatim}
>>> x = arange(11)
array([ 0, 1, 2, 3, 4, 5, 6, 7, 8, 9, 10])

\end{verbatim}
\end{framed}
%----------------------------------%
\subsubsection{\texttt{meshgrid}}
\texttt{meshgrid} broadcasts two vectors to produce two 2-dimensional arrays, and is a useful function when plotting
3-dimensional functions.
\begin{framed}
\begin{verbatim}
>>> x = arange(5)
>>> y = arange(3)
>>> X,Y = meshgrid(x,y)
>>> X
array([[0, 1, 2, 3, 4],
[0, 1, 2, 3, 4],
[0, 1, 2, 3, 4]])
>>> Y
array([[0, 0, 0, 0, 0],
[1, 1, 1, 1, 1],
[2, 2, 2, 2, 2]])
\end{verbatim}
\end{framed}
%-----------------------------------------------%
\newpage
\subsubsection{\texttt{r\_}}
\texttt{r\_} is a convenience function which generates 1-dimensional arrays from slice notation. While \texttt{r\_} is highly
flexible, the most common use is 
\begin{verbatim} r_[ start : end : stepOrCount ] 
\end{verbatim}
where \texttt{start} and \texttt{end} are the start and end
points, and \texttt{stepOrCount} can be either a step size, if a real value, or a count, if complex.
\begin{framed}
\begin{verbatim}
>>> # arange equiv
>>> r_[0:10:1] 
array([0, 1, 2, 3, 4, 5, 6, 7, 8, 9])
>>>
>>> # arange equiv
>>> r_[0:10:.5] 
array([ 0. , 0.5, 1. , 1.5, 2. , 2.5, 3. , 3.5, 4. , 4.5, 5. ,
5.5, 6. , 6.5, 7. , 7.5, 8. , 8.5, 9. , 9.5])
>>>
>>> # linspace equiv, includes end point
>>> r_[0:10:5j] 
array([ 0. , 2.5, 5. , 7.5, 10. ])
\end{verbatim}
\end{framed}
\noindent \texttt{r\_} can also be used to concatenate slices using commas to separate slice notation blocks.
\begin{framed}
\begin{verbatim}
>>> r_[0:2, 7:11, 1:4]
array([ 0, 1, 7, 8, 9, 10, 1, 2, 3])
\end{verbatim}
\end{framed}
Note that r\_ is not a function and that is used with [].
%----------------------------------%
\subsubsection{\texttt{c\_}}
\texttt{c\_} is virtually identical to \texttt{r\_} except that column arrays are generates, which are 2-dimensional (second
dimension has size 1).
\begin{framed}
\begin{verbatim}
>>> c_[0:5:2]
array([[0],[2],[4]])
>>>
>>> c_[1:5:4j]
array([[ 1. ],
[ 2.33333333],
[ 3.66666667],
[ 5. ]])
\end{verbatim}
\end{framed}
\texttt{c\_}, like \texttt{r\_}, is not a function and is used with [].

%=============================================================================%
\subsubsection{\texttt{ix\_}}
\texttt{ix\_(a,b)} constructs an n-dimensional open mesh from n 1-dimensional lists or arrays. The output of
\texttt{ix\_ }is an n-element tuple containing 1-dimensional arrays. The primary use of \texttt{ix\_} is to simplify selecting
slabs inside a matrix. Slicing can also be used to select elements froman array as long as the slice pattern is
regular. 

\noindent \texttt{ix\_ }is particularly useful for selecting elements froman array using indices which are not regularly
spaced, as in the final example.
\begin{framed}
\begin{verbatim}
>>> x = reshape(arange(25.0),(5,5))
>>> x
array([[ 0., 1., 2., 3., 4.],
[ 5., 6., 7., 8., 9.],
[ 10., 11., 12., 13., 14.],
[ 15., 16., 17., 18., 19.],
[ 20., 21., 22., 23., 24.]])
>>>
>>> x[ix_([2,3],[0,1,2])] # Rows 2 & 3, cols 0, 1 and 2
array([[ 10., 11., 12.],
[ 15., 16., 17.]])
>>>
>>> x[2:4,:3] # Same, standard slice
array([[ 10., 11., 12.],
[ 15., 16., 17.]])
>>>
>>> x[ix_([0,3],[0,1,4])] # No slice equiv
\end{verbatim}
\end{framed}
%=====================%
\newpage
\subsubsection{\texttt{mgrid}}
\texttt{mgrid }is very similar to meshgrid but behaves like\texttt{ r\_} and \texttt{c\_} in that it takes slices as input, and uses a
real valued variable to denote step size and complex to denote number of values. The output is an n + 1
dimensional vector where the first index of the output indexes the meshes.
\begin{framed}
\begin{verbatim}
>>> mgrid[0:3,0:2:.5]
>>>
>>> mgrid[0:3:3j,0:2:5j]
>>>
\end{verbatim}
\end{framed}

\subsubsection{\texttt{ogrid}}
\texttt{ogrid} is identical to mgrid except that the arrays returned are always 1-dimensional. \texttt{ogrid} output is generally
more appropriate for looping code, while \texttt{mgrid} is usually more appropriate for vectorized code.
When the size of the arrays is large, then \texttt{ogrid} uses much less memory.
\begin{framed}
\begin{verbatim}
>>> ogrid[0:3,0:2:.5]
[array([[ 0.],
[ 1.],
[ 2.]]), array([[ 0. , 0.5, 1. , 1.5]])]
>>> ogrid[0:3:3j,0:2:5j]
[array([[ 0. ],
[ 1.5],
[ 3. ]]),
array([[ 0. , 0.5, 1. , 1.5, 2. ]])]
\end{verbatim}
\end{framed}
\end{document}
