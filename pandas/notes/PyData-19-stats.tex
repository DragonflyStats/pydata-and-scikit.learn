%	19.1 Simulating Random Variables	
%		19.1.1 Core Random Number Generators
%		19.1.2 Random Array Functions
%		19.1.3 Select Random Number Generators
%	19.2 Simulation and Random Number Generation	
%		19.2.1 State
%		19.2.2 Seed
%		19.2.3 Replicating Simulation Data
%	19.3 Statistics Functions	
%	19.4 Continuous Random Variables	
%		19.4.1 Example: gamma
%		19.4.2 Important Distributions
%		19.4.3 Frozen Random Variable Object
%	19.5 Select Statistics Functions	
%	19.6 Select Statistical Tests	

\documentclass[Pydata.tex]{subfiles} 
\begin{document} 
	\large
\section{Probability Functions}


\subsection{Random Number Generation with NumPy}
\begin{framed}
\noindent For the sake of brevity, the specific functions names are given in the example below, rather than the full specification.
\\
The \texttt{rand()} command is fully specified as \texttt{np.random.rand()}

\end{framed}
% 19.1 Simulating Random Variables
% - 19.1.1 Core Random Number Generators
NumPy random number generators are all stored in the module \texttt{numpy.random}. 
These can be imported with using \texttt{import numpy as np} and then calling \texttt{np.random.rand}, for example, or 
by importing \texttt{import numpy.random as rnd} and \texttt{using rnd.rand.1}.
\subsubsection{\texttt{rand}, \texttt{random\_sample}}
\texttt{rand} and \texttt{random\_sample }are uniform randomnumber generators whichare
 identicalexceptthat rand takes a variable number 
of integer inputs – one for each dimension – while \texttt{random\_sample} takes a n-element tuple. 

\noindent \texttt{random\_sample} is the preferred NumPy function, and \texttt{rand} is a convenience function primarily for \textit{MATLAB} users.
\begin{framed}
\begin{verbatim}
x = rand(3,4,5) 
y = random_sample((3,4,5))
\end{verbatim}
\end{framed}
%===============================================%

% -------19.1 Simulating Random Variables
% -------19.1.1 Core Random Number Generators

\subsubsection{\texttt{randn, standard\_normal}}
\texttt{randn} and \texttt{standard\_normal} are standard norma (i.e. Z-value)l random number generators. \texttt{randn}, like \texttt{rand}, takes a
variable number of integer inputs, and \texttt{standard\_normal} takes an n-element tuple. Both can be called
with no arguments to generate a single standard normal (e.g. randn()). \texttt{standard\_normal }is the preferred
NumPy function, and randn is a convenience function primarily for MATLAB users .
\begin{framed}
\begin{verbatim}
>>> x = randn(3,4,5)
>>> y = standard_normal((3,4,5))
\end{verbatim}
\end{framed}
\subsubsection{\texttt{randint, random\_integers}}
\texttt{randint} and \texttt{random\_integers} are uniform integer random number generators which take 3 inputs: low,
high and size. 
\begin{itemize}
\item \texttt{low} is the lower bound of the integers generated, 
\item \texttt{high} is the upper,
\item \texttt{size} is a n-elementtuple. 
\end{itemize}

\noindent \textbf{Important:}\\
 \texttt{randint} and \texttt{random\_integers} differ in that \texttt{randint} generates integers exclusive of the value in \texttt{high}
(as do most Python functions), while \texttt{random\_integers} includes the value in \texttt{high} in its range.
\begin{framed}
\begin{verbatim}
x = randint(0,10,(100))
x.max() # Is 9 since range is [0,10)
y = random_integers(0,10,(100))
 y.max() # Is 10 since range is [0,10]
\end{verbatim}
\end{framed}
% -------19.1.2 Random Array Functions
\subsubsection{\texttt{shuffle}}
\texttt{shuffle} randomly reorders the elements of an array in place.
\begin{framed}
\begin{verbatim}
>>> x = arange(10)
>>> shuffle(x)
>>> x
array([4, 6, 3, 7, 9, 0, 2, 1, 8, 5])
\end{verbatim}
\end{framed}
\subsubsection{\texttt{permutation}}
permutation returns randomly reordered elements of an array as a copy while not directly changing the
input.
\begin{framed}
\begin{verbatim}
>>> x = arange(10)
>>> permutation(x)
array([2, 5, 3, 0, 6, 1, 9, 8, 4, 7])
>>> x
array([0, 1, 2, 3, 4, 5, 6, 7, 8, 9])
\end{verbatim}
\end{framed}
% -------19.1.3 Select Random Number Generators
NumPy provides a large selection of random number generators for specific distribution. All take between
0 and 2 required inputs which are parameters of the distribution, plus a tuple containing the size of the
output. All random number generators are in the module numpy.random.
%

% 19.1.2 Random Array Functions
%======================================================================== %
% 19.1.3 Select Random Number Generators
%\subsubsection{Select Random Number Generators}
%% NumPyprovidesalargeselectionofrandomnumbergeneratorsforspecificdistribution. 
%
%All takebetween 0 and 2 required inputs which are parameters of the distribution, plus a tuple containing the size of the output. Allrandomnumbergeneratorsareinthemodule 
%numpy.random.

% \subsection{Bernoulli}
% ThereisnoBernoulligenerator. Instead usebinomial(1,p)to generateasingle draworbinomial(1,p,(10,10)) togenerate anarray where % p istheprobabilityofsuccess.

\newpage
%=================================%

%19.2 Simulation and Random Number Generation
\subsection{Simulation and Random Number Generation}
\begin{itemize}
\item Computer simulated random numbers are usually constructed from very complex but ultimately deterministic
functions. 
\item Because they are not actually random, simulated random numbers are generally described
to as \textbf{pseudo-random}. 
\item All pseudo-random numbers in NumPy use one core random number generator
based on the \textbf{\textit{Mersenne Twister}}, a generator which can produce a very long series of pseudo-random
data before repeating (up to $2^19937 - 1$ non-repeating values).
\end{itemize}

\subsubsection{\texttt{RandomState}}
\texttt{RandomState} is the class used to control the random number generators. Multiple generators can be initialized
by RandomState.
\begin{framed}
	\begin{verbatim}
	>>> gen1 = np.random.RandomState()
	>>> gen2 = np.random.RandomState()
	>>> gen1.uniform() # Generate a uniform
	0.6767614077579269
	>>> state1 = gen1.get_state()
	>>> gen1.uniform()
	0.6046087317893271
	>>> gen2.uniform() # Different, since gen2 has different seed
	0.04519705909244154
	>>> gen2.set_state(state1)
	>>> gen2.uniform() # Same uniform as gen1 after assigning state
	0.6046087317893271
	\end{verbatim}
\end{framed}
%==============%
% 19.2.1
\subsubsection{\texttt{ State}}
Pseudo-random number generators track a set of values known as the \textit{state}. The state is usually a vector
which has the property that if two instances of the same pseudo-random number generator have the
same state, the sequence of pseudo-random numbers generated will be identical. The state of the default
random number generator can be read using \texttt{numpy.random.get\_state} and can be restored using
\texttt{numpy.random.set\_state}.
\begin{framed}
	\begin{verbatim}>>> st = get_state()
	>>> randn(4)
	array([ 0.37283499, 0.63661908, 1.51588209,
	1.36540624])
	>>> set_state(st)
	>>> randn(4)
	array([ 0.37283499, 0.63661908, 1.51588209,
	1.36540624])
	\end{verbatim}
\end{framed}
The two sequences are identical since they the state is the same when \texttt{randn} is called. The state is a 5-
element tuple where the second element is a 625 by 1 vector of unsigned 32-bit integers. In practice the
state should only be stored using get\_state and restored using \texttt{set\_state}.
\newpage

\subsubsection{\texttt{get\_state}}
%19.2
\texttt{get\_state()} gets the current state of the random number generator, which is a 5-element tuple. It can be
called as a function, in which case it gets the state of the default random number generator, or as a method
on a particular instance of RandomState.
\subsubsection*{\texttt{set\_state}}
%19.2.1
\texttt{set\_state(state)} sets the state of the random number generator. It can be called as a function, in which
case it sets the state of the default random number generator, or as a method on a particular instance of
RandomState.\\ \texttt{set\_state} should generally only be called using a state tuple returned by \texttt{get\_state}.

\subsubsection*{\texttt{seed}}
% 19.2.2 
\texttt{numpy.random.seed} is a more useful function for initializing the random number generator, and can be
used in one of two ways. \texttt{seed()} will initialize (or reinitialize) the random number generator using some
actual random data provided by the operating system.

\texttt{seed( s )} takes a vector of values (can be scalar) to
initialize the random number generator at particular state. seed( s ) is particularly useful for producing
simulation studies which are reproducible. \\
\\ In the following example, calls to \texttt{seed()} produce different
random numbers, since these reinitialize using random data from the computer, while calls to \texttt{seed(0)}
produce the same (sequence) of random numbers.
\begin{framed}
\begin{verbatim}
>>> seed()
>>> randn()
array([ 0.62968838])
>>> seed()
>>> randn()
array([ 2.230155])
>>> seed(0)
>>> randn()
array([ 1.76405235])
>>> seed(0)
>>> randn()
array([ 1.76405235])
\end{verbatim}
\end{framed}
NumPy always calls \texttt{seed()}  when the first randomnumberis generated. As a result. calling \texttt{standard\_normal()}
across two “fresh” sessions will not produce the same random number.

%=========================================================%
%19.3 - Statistical Functions

% - Median
% - Standard Deviation
% - Variance
% - Correlation
% - Covariance
% - Histograms
% - Histogram Plots
%=========================================================%
% Page 228
\newpage
\subsection{Probability Distributions}
\subsubsection{\texttt{normal}}
\texttt{normal()} generates a set of random numbers from a standard Normal (Gaussian).\\ \texttt{normal(mu, sigma)} generates draws from
a Normal distribution with mean $\mu$ and standard deviation $\sigma$. \texttt{normal(mu, sigma, (10,10))} generates a 10 by 10 array
of draws from a Normal with mean $\mu$ and standard deviation $\sigma$. 

\texttt{normal(mu, sigma)} is equivalent to \texttt{mu + sigma $\ast$ \texttt{standard\_normal()}}.
\subsubsection{\texttt{poisson}}
\texttt{poisson()} generates a set of random numbers from a Poisson distribution with $\lambda = 1$.\\ \texttt{poisson(lambda)} generates a draw
from a Poisson distribution with expectation $\lambda$. poisson(lambda, (10,10)) generates a 10 by 10 array of
draws from a Poisson distribution with expectation $\lambda$.
\subsubsection{\texttt{standard\_t}}
\texttt{standard\_t(nu)} generates a set of random numbers from a Student’s t with shape parameter $\nu$.\\ \texttt{standard\_t(nu, (10,10))}
generates a 10 by 10 array of draws from a Student’s t with shape parameter $\nu$.
\subsubsection{\texttt{uniform}}
\texttt{uniform()} generates a uniform random variable on (0, 1). \\ \texttt{uniform(low, high)} generates a uniform on
(l , h). \texttt{uniform(low, high, (10,10))} generates a 10 by 10 array of uniforms on (l , h).
%=========================================================%
\newpage

%========================================================================= %
% 19.4 Continuous Random Variables

\subsection{Continuous Random Variables}

SciPy contains a large number of functions for working with continuous random variables. Each function
resides in its own class (e.g. norm for Normal or gamma for Gamma), and classes expose methods for random
number generation, computing the PDF, CDF and inverse CDF, fitting parameters using MLE, and
computing various moments. The methods are listed below, where dist is a generic placeholder for the
distribution name in SciPy. 
%---------------------------%
\begin{itemize} 
	\item \texttt{dist.rvs}\\
	Pseudo-randomnumbergeneration. Generically, rvs is called using dist.rvs(*args, loc=0, scale=1, size=size)
	where size is an n-element tuple containing the size of the array to be generated.
	\item \texttt{dist.pdf}\\
	Probability density function evaluation for an array of data (element-by-element). Generically, pdf is
	called using \texttt{dist.pdf(x, *args, loc=0, scale=1)} where x is an array that contains the values to use when
	evaluating PDF.
	\item \texttt{dist.logpdf}\\
	Log probability density function evaluation for an array of data (element-by-element). Generically, logpdf
	is called using dist.logpdf(x, *args, loc=0, scale=1) where x is an array that contains the values to use
	when evaluating log PDF.
	\item \texttt{dist.cdf}\\
	Cumulative distribution function evaluation for an array of data (element-by-element). Generically, cdf
	is called using d\texttt{ist.cdf(x, *args, loc=0, scale=1)} where x is an array that contains the values to use
	when evaluating CDF.
	\item \texttt{dist.ppf}\\
	Inverse CDF evaluation (also known as percent point function) for an array of values between 0 and 1.
	Generically, ppf is called using \texttt{dist.ppf(p, *args, loc=0, scale=1)} where p is an array with all elements
	between 0 and 1 that contains the values to use when evaluating inverse CDF.
	\item \texttt{dist.fit}\\
	Estimate shape, location, and scale parameters from data by maximum likelihood using an array of data.
	\\ Generically, fit is called using \texttt{dist.fit(data, *args, floc=0, fscale=1)} where data is a data array used
	to estimate the parameters. \\ floc forces the location to a particular value (e.g. floc=0). \texttt{fscale} similarly
	forces the scale to a particular value (e.g. \texttt{fscale=1}) .\\  It is necessary to use floc and/or fscale when
	computing MLEs if the distribution does not have a location and/or scale.\\ For example, the gamma distribution
	is defined using 2 parameters, often referred to as shape and scale. \\ In order to useMLto estimate
	parameters from a gamma, floc=0 must be used.
	\item \texttt{dist.median}\\
	Returns the median of the distribution. Generically, median is called using dist.median(*args, loc=0, scale=1).
	\item \texttt{dist.mean}\\
	Returns the mean of the distribution. Generically, mean is called using dist.mean(*args, loc=0, scale=1).
	\item\texttt{ dist.moment}\\
	nth non-centralmomentevaluation of the distribution. Generically, moment is called using dist.moment(r, *args,
	loc=0, scale=1) where r is the order of the moment to compute.
	\item \texttt{dist.varr}\\
	Returns the variance of the distribution. Generically, var is called using dist.var(*args, loc=0, scale=1).
	\item \texttt{dist.std}\\
	Returns the standard deviation of the distribution. Generically, std is called using dist.std(*args, loc=0, scale=1).
\end{itemize}
%------------------------------------------------------------------%
\newpage %19.4.1 Example: gamma
\subsubsection{Example}
The gamma distribution is used as an example. \\ The gamma distribution takes 1 shape parameter a (a is
the only element of *args), which is set to 2 in all examples.
\begin{framed}
	\begin{verbatim}
		
		
		>>> import scipy.stats as stats
		>>> gamma = stats.gamma
		>>> gamma.mean(2), gamma.median(2), gamma.std(2), gamma.var(2)
		(2.0, 1.6783469900166608, 1.4142135623730951, 2.0)
		>>> gamma.moment(2,2) gamma.
		moment(1,2)**2 # Variance

		>>> gamma.cdf(5, 2), gamma.pdf(5, 2)
		(0.95957231800548726, 0.033689734995427337)
		>>> gamma.ppf(.95957231800548726, 2)
		5.0000000000000018

		>>> log(gamma.pdf(5, 2)) gamma.
		logpdf(5, 2)
		0.0

		>>> gamma.rvs(2, size=(2,2))
		array([[ 1.83072394, 2.61422551],
		[ 1.31966169, 2.34600179]])
		>>> gamma.fit(gamma.rvs(2, size=(1000)), floc = 0) # a, 0, shape
		(2.209958533078413, 0, 0.89187262845460313)
	\end{verbatim}
\end{framed}
\end{document}