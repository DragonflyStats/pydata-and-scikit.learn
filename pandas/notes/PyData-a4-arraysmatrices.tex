%	4.1 Array	
%		4.1.1 Array dtypes
%	4.2 Matrix	
%	4.3 1-dimensional Arrays
%	4.4 2-dimensional Arrays	GOOD
%	4.5 Multidimensional Arrays	 GOOD
%	4.6 Concatenation	MORE
%	4.7 Accessing Elements of an Array	
%		4.7.1 Scalar Selection
%		4.7.2 Array Slicing
%		4.7.3 Mixed Selection using Scalar and Slice Selectors
%		4.7.4 Assignment using Slicing
%		4.7.5 Linear Slicing using flat
%	4.8 Slicing and Memory Management	
%	4.9 import and Modules	
%	4.10 Calling Functions	
%--------------------------------------------------------------------------------------------%


\documentclass[Pydata.tex]{subfiles} 
\begin{document} 
\section{Arrays and Matrices}
\textit{\textbf{NumPy}} provides the most important data types for econometrics, statistics and numerical analysis – \textbf{arrays}
and \textbf{matrices}. The difference between these two data types are:
\begin{itemize}
\item Arrays can have 1, 2, 3 or more dimensions, and matrices \textbf{always} have 2 dimensions. This means
that a 1 by n vector stored as an array has 1 dimension and n elements, while the same vector stored
as a matrix has 2-dimensions where the sizes of the dimensions are 1 and n (in either order).
\item  Standard mathematical operators on arrays operate element-by-element. This is not the case for
matrices, where multiplication ($\ast$) follows the rules of \textit{linear algebra}.
\item 2-dimensional arrays can be
multiplied using the rules of linear algebra using dot. 
\item Similarly, the function multiply can be used
on two matrices for \textit{element-by-element} multiplication.
\item  Arrays are more common than matrices, and all functions are thoroughly tested with arrays. 
\item Functions
should also work with matrices, but an occasional strange result may be encountered.
\item  Arrays can be quickly treated as a matrix using either \texttt{asmatrix} or \texttt{mat} without copying the underlying
data.
\end{itemize}
The best practice is to use arrays and to use the \texttt{asmatrix} view when writing linear algebra-heavy code. It is
also important to test any custom function with both arrays and matrices to ensure that false assumptions
about the behavior of multiplication have not been made.

\subsection{Arrays}
Arrays are the base data type in NumPy, are are arrays in some ways similar to lists since they both contain
collections of elements. The focus of this section is on \textit{homogeneous} arrays containing numeric data
– that is, an array where all elements have the same numeric type

%(heterogeneous arrays are covered in KS Chapters 16 and 17). 

Additionally, arrays, unlike lists, are always rectangular so that all rows have the same
number of elements.

Arrays are initialized from lists (or tuples) using array. Two-dimensional arrays are initialized using
lists of lists (or tuples of tuples, or lists of tuples, etc.), and higher dimensional arrays can be initialized by
further nesting lists or tuples.

\begin{framed}
\begin{verbatim}
>>> x = [0.0, 1, 2, 3, 4]
>>> y = array(x)
>>> y
array([ 0., 1., 2., 3., 4.])
>>> type(y)
numpy.ndarray
\end{verbatim}
\end{framed}
Two (or higher) -dimensional arrays are initialized using nested lists.
\begin{framed}
\begin{verbatim}
>>> y = array([[0.0, 1, 2, 3, 4], [5, 6, 7, 8, 9]])
>>> y
array([[ 0., 1., 2., 3., 4.],
[ 5., 6., 7., 8., 9.]])
>>> shape(y)
(2L, 5L)
>>> y = array([[[1,2],[3,4]],[[5,6],[7,8]]])
>>> y
array([[[1, 2],
[3, 4]],
[[5, 6],
[7, 8]]])
>>> shape(y)
(2L, 2L, 2L)
\end{verbatim}
\end{framed}

\subsubsection*{Array dtypes}
%4.1.1.
Homogeneous arrays can contain a variety of numeric data types. The most useful is ’\texttt{float64}’, which corresponds
to the python built-in data type of float (andC/C++double). By default, calls to array will preserve
the type of the input, if possible. If an input contains all integers, it will have a dtype of ’int32’ (similar to
the built in data type int). If an input contains integers, floats, or a mix of the two, the array’s data type will
be float64. If the input contains a mix of integers, floats and complex types, the array will be initialized
to hold complex data.
\begin{framed}
\begin{verbatim}
>>> x = [0, 1, 2, 3, 4] # Integers
>>> y = array(x)
>>> y.dtype
dtype(’int32’)
>>> x = [0.0, 1, 2, 3, 4] # 0.0 is a float
>>> y = array(x)
>>> y.dtype
dtype(’float64’)
>>> x = [0.0 + 1j, 1, 2, 3, 4] # (0.0 + 1j) is a complex
>>> y = array(x)
>>> y
array([ 0.+1.j, 1.+0.j, 2.+0.j, 3.+0.j, 4.+0.j])
>>> y.dtype
dtype(’complex128’)
\end{verbatim}
\end{framed}
NumPy attempts to find the smallest data type which can represent the data when constructing an array.
It is possible to force NumPy to select a particular dtype by using the keyword argument dtype=datetype
when initializing the array.
\begin{framed}
\begin{verbatim}
>>> x = [0, 1, 2, 3, 4] # Integers
>>> y = array(x)
>>> y.dtype
dtype(’int32’)
>>> y = array(x, dtype=’float64’) # String dtype
>>> y.dtype
dtype(’float64’)
>>> y = array(x, dtype=float64) # NumPy type dtype
>>> y.dtype
dtype(’float64’)
\end{verbatim}
\end{framed}
\newpage
%=======================================%
\subsection{Matrix}
Matrices are essentially a subset of arrays, and behave in a virtually identical manner. The two important
differences are:

\begin{itemize}
\item Matrices always have 2 dimensions
\item Matrices follow the rules of linear algebra for Matrix Multiplication ($\ast$)
\end{itemize}

\begin{framed}
\begin{verbatim}
>>> x = [0.0, 1, 2, 3, 4] # Any float makes all float
>>> y = array(x)
>>> type(y)
numpy.ndarray
>>> y * y # Elementbyelement
array([ 0., 1., 4., 9., 16.])
>>> z = asmatrix(x)
>>> type(z)
numpy.matrixlib.defmatrix.matrix
>>> z * z # Error
ValueError: matrices are not aligned
\end{verbatim}
\end{framed}
\newpage
\subsection{1-dimensional Arrays}
%4.3 - ALL good so far
A vector
\[x = [1 2 3 4 5]\]
is entered as a 1-dimensional array using
\begin{framed}
\begin{verbatim}
>>> x=array([1.0,2.0,3.0,4.0,5.0])
array([ 1., 2., 3., 4., 5.])
>>> ndim(x)
1
\end{verbatim}
\end{framed}
%----------------------%
If an array with 2-dimensions is required, it is necessary to use a trivial nested list.
\begin{framed}
\begin{verbatim}
>>> x=array([[1.0,2.0,3.0,4.0,5.0]])
array([[ 1., 2., 3., 4., 5.]])
>>> ndim(x)
2
\end{verbatim}
\end{framed}

A matrix is always 2-dimensional and so a nested list is not required to initialize a a row matrix
\begin{framed}
\begin{verbatim}
>>> x=matrix([1.0,2.0,3.0,4.0,5.0])
>>> x
matrix([[ 1., 2., 3., 4., 5.]])
>>> ndim(x)
2
\end{verbatim}
\end{framed}
Notice that the output matrix representation uses nested lists ([[ ]]) to emphasize the 2-dimensional
structure of all matrices. The column vector,
\[x =
\left[ 
\begin{array}{c}
1 \\
2 \\
3 \\
4 \\
5
\end{array}
\right]
\]
is entered as a matrix or 2-dimensional array using a set of nested lists
\begin{framed}
\begin{verbatim}
>>> x=matrix([[1.0],[2.0],[3.0],[4.0],[5.0]])
>>> x
matrix([[ 1.],
[ 2.],
[ 3.],
[ 4.],
[ 5.]])
>>> x = array([[1.0],[2.0],[3.0],[4.0],[5.0]])
>>> x
array([[ 1.],
[ 2.],
[ 3.],
[ 4.],
[ 5.]])
\end{verbatim}
\end{framed}
\newpage
%=======================================%




%-------------------------------------------------------------------------%

\newpage
% KS 4.4 2-dimensional Arrays
% GOOD
\subsection{2-dimensional Arrays}
Matrices and 2-dimensional arrays are rows of columns, and so
\[x =
\begin{array}{ccc}
1 & 2 & 3 \\
4 & 5 & 6 \\
7 & 8 & 9 \\
\end{array}
\]
is input by enter the matrix one row at a time, each in a list, and then encapsulate the row lists in another
list.
\begin{framed}
\begin{verbatim}
>>> x = array([[1.0,2.0,3.0],[4.0,5.0,6.0],[7.0,8.0,9.0]])
>>> x
array([[ 1., 2., 3.],
[ 4., 5., 6.],
[ 7., 8., 9.]])
\end{verbatim}
\end{framed}

%---------------------------------------------%
% KS - 4.5 Multidimensional Arrays
% GOOD
\newpage
\subsection{Multidimensional Arrays}
\begin{itemize}
\item Higher dimensional arrays are useful when tracking matrix valued processes through time, such as a time varying
covariance matrices. 
\item Multidimensional (N -dimensional) arrays are available for N up to about 30,
\item depending on the size of each matrix dimension. 
\item Manually initializing higher dimension arrays is tedious
and error prone, and so it is better to use functions such as \texttt{zeros((2, 2, 2))} or \texttt{empty((2, 2, 2))}.
\end{itemize}
%---------------------------------------------%

%\subsection{Concatenation}
% KS -  4.6 Concatenation
% Lots to add in
% Concatenation is the process by which one vector or matrix is appended to another. Arrays and matrices
% can be concatenation horizontally or vertically
%---------------------------------------------%
% KS 4.7 Accessing Elements of an Array
\subsection{Accessing Elements of an Array }


% Numerical indexing and logical indexing both depends on specialized functions and
% so these methods are discussed in Chapter 12

\begin{itemize}
\item Four methods are available for accessing elements contained within an array: scalar selection, slicing,
numerical indexing and logical (or Boolean) indexing. 
\item Scalar selection and slicing are the simplest and so
are presented first. 
%\item Numerical indexing and logical indexing both depends on specialized functions and
%so these methods are discussed in Chapter 12.
\end{itemize}
%-------------------------------------------------------------------------%
% KS page 59
% KS - 4.9 import and Modules
\newpage
\subsection{The \texttt{import} function}
\begin{itemize}
\item Python, by default, only has access to a small number of built-in types and functions. The vast majority of
functions are located in modules, and before a function can be accessed, the module which contains the
function must be imported. 

\item For example, when using \texttt{ipython --pylab} (or any variants), a large number
of modules are automatically imported, including NumPy and matplotlib.
\item This is style of importing useful
for learning and interactive use, but care is needed to make sure that the correct module is imported when
designing more complex programs.

\item \texttt{import} can be used in a variety of ways. The simplest is to use from module \texttt{import *} which imports
all functions in module. 
\item This method of using import can dangerous since if you use it more than once,
it is possible for functions to be hidden by later imports. 
\item A better method is to just import the required
functions. 
\item This still places functions at the top level of the namespace, but can be used to avoid conflicts.
\begin{framed}
\begin{verbatim}
from pylab import log2 # Will import log2 only
from scipy import log10 # Will not import the log2 from SciPy
\end{verbatim}
\end{framed}
%KS page 59
\item The only difference between these two is that \texttt{import scipy} is implicitly calling \texttt{import scipy as scipy}.
\item When this formof import is used, functions are used with the “as” name. For example, the load provided
by NumPy is accessed using \texttt{sp.log2}, while the pylab load is \texttt{pl.log2} – and both can be used where appropriate.
\item While this method is the most general, it does require slightly more typing.
\end{itemize}
%--------------------------------------------------------------------------------------------%
\end{document}
