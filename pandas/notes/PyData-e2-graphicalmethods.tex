\documentclass[KSmain.tex]{subfiles} 
\begin{document} 
	\newpage
	\section*{Graphical Methods}
\subsection*{Histograms}
\begin{itemize}
	\item Histograms can be produced using \texttt{hist}. 
	\item A basic histogram produced using the code below is presented
	below.
	
	\item  This example sets the number of bins used in producing the histogram using the
	keyword argument \texttt{bins}.
\end{itemize}
\subsection*{Adding a Title and Legend}
Titles are added with title and legends are added with legend. \texttt{legend} requires that lines have labels,
which is why 3 calls are made to plot – each series has its own label.
% Executing the next code block producesa the image in figure 15.8, panel (a).

\begin{framed}
	\begin{verbatim}
 x = cumsum(randn(100,3), axis = 0)
 
 plot(x[:,0],’b’,label = ’Series 1’)
 hold(True)
 plot(x[:,1],’g.’,label = ’Series 2’)
 plot(x[:,2],’r:’,label = ’Series 3’)
 
 legend()
 title(’Basic Legend’)
	\end{verbatim}
\end{framed}

\begin{itemize}
\item \texttt{legend} takes keyword arguments which can be used to change its location (loc and an integer, see the
docstring), remove the frame (frameon) and add a title to the legend box (title). 
\item The output of a simple
example using these options is presented in panel (b).
\end{itemize}
%============================================================================== %

\begin{framed}
	\begin{verbatim}
	 plot(x[:,0],’b’,
	 label = ’Series 1’)
	 hold(True)
	 plot(x[:,1],’g.’,label = ’Series 2’)
	 plot(x[:,2],’r:’,label = ’Series 3’)
	 legend(loc = 0, frameon = False, title = ’The Legend’)
	 title(’Improved Legend’)
	\end{verbatim}
\end{framed}

\newpage
\subsection{Plotting}
% - http://nbviewer.ipython.org/github/twiecki/financial-analysis-python-tutorial/blob/master/1.%20Pandas%20Basics.ipynb

\begin{framed}
	\begin{verbatim}
	close_px.plot(label='AAPL')
	mavg.plot(label='mavg')
	plt.legend()
	
	\end{verbatim}
\end{framed}

\end{document}