\documentclass[Pydata.tex]{subfiles} 
\begin{document} 
%-------------------------------------------------------------------------%
% KS page 59
% KS - 4.9 import and Modules
\newpage
\Large\section{The \texttt{import} function}
\begin{itemize}
\item Python, by default, only has access to a small number of built-in types and functions. The vast majority of
functions are located in modules, and before a function can be accessed, the module which contains the
function must be imported. 

\item For example, when using \texttt{ipython --pylab} (or any variants), a large number
of modules are automatically imported, including \textbf{\textit{NumPy}} and \textbf{\textit{matplotlib}}.
\item This is style of importing useful
for learning and interactive use, but care is needed to make sure that the correct module is imported when
designing more complex programs.

\item \texttt{import} can be used in a variety of ways. The simplest is to use from module \texttt{import *} which imports
all functions in module.

\item You can give each module an ``alias" too, using the \texttt{as} specifier.
\end{itemize}

\begin{framed}
\begin{verbatim}
import pandas as pd
import numpy as np
import seaborn as sb
\end{verbatim}	
\end{framed}	
	
\subsubsection*{Caution}
\begin{itemize}
 
\item This method of using import can dangerous since if you use it more than once,
it is possible for functions to be hidden by later imports. 
\item A better method is to just import the required
functions. 
\item This still places functions at the top level of the namespace, but can be used to avoid conflicts.
\begin{framed}
\begin{verbatim}
from pylab import log2 # Will import log2 only
from scipy import log10 # Will not import the log2 from SciPy
\end{verbatim}
\end{framed}
%KS page 59
\item The only difference between these two is that \texttt{import scipy} is implicitly calling \texttt{import scipy as scipy}.
\item When this formof import is used, functions are used with the “as” name. For example, the load provided
by NumPy is accessed using \texttt{sp.log2}, while the pylab load is \texttt{pl.log2} – and both can be used where appropriate.
\item While this method is the most general, it does require slightly more typing.
\end{itemize}
%--------------------------------------------------------------------------------------------%
\end{document}